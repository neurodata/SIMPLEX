
\section{Personnel, Qualifications, and Commitments}

% \jason{ can you please put youngser, nam, and carey's statements into the same format that randal used, it is the standard, thanks.}


% \emph{List key personnel (no more than one page per person), showing a concise summary of their qualifications, discussion of previous accomplishments, and work in this or closely related research areas. Indicate the level of effort in terms of hours to be expended by each person during each contract year and other (current and proposed) major sources of support for them and/or commitments of their efforts. DARPA expects all key personnel associated with a proposal to make substantial time commitment to the proposed activity and the proposal will be evaluated accordingly. It is DARPA’s intention to put key personnel conditions into the awards, so proposers should not propose personnel that are not anticipated to execute the award.}


\para{Joshua T.~Vogelstein} received a BS degree in Biomedical Engineering from Washington University in St. Louis in 2002, a MS in Applied Mathematics and Statistics (AMS) and a PhD in Neuroscience in 2009 from Johns Hopkins University. He then spent a couple years as a Post-Doctoral Fellow with Carey Priebe, followed by a brief appointment as Research Faculty in AMS, and then to Duke University for another Research Faculty appointment at Duke's big data center.  He recently came back home to Johns Hopkins University, where he is an Assistant Professor in Biomedical Engineering, as well as core faculty at the Institute for Computational Medicine and the Center for Imaging Science, with joint appointments in the works for Computer Science, Neuroscience, and Biostatistics.  He is also a member of the Institute for Data Intensive Engineering and Sciences and the Human Language Technology Center of Excellence.  He has published in a wide variety of top venues, ranging from Science and Science Translation Medicine, and Nature Methods, to Annals of Applied Statistics and Neural Information Processing Systems,   to Neuron, Nature Neuroscience, and Journal of Neurophysiology.

As evidenced from his list of publications, Dr. Vogelstein works extensively on all aspects of this proposal, from mathematical theory to machine learning and signal processing, to neuroscience.


\vspace{10pt}

\subpara{Levels of Effort}  Dr. Vogelstein will commit 3 months (25\%) of effort to this grant per year.  He has the following sources of funding, noted with months of effort and project role.
\small

\vspace{-5pt}

\begin{list}{[\arabic{enumi}]}{\settowidth{\labelwidth}{0.72 mos., co-PI}\leftmargin\labelwidth\advance\leftmargin 0.2in \setlength{\labelsep}{0.2in}}
\addtolength{\itemsep}{-4pt}

\item[3.0, co-PI] NIH PAR-12-806 (NIBIB), CRCNS: Data Sharing: The EM Open Connectome Project. 
09/01/12--08/31/15.

\item[3.0, co-PI] DARPA N66001-14-1-4028, Scalable Brain Graph Analyses using Big-Memory High-IOPS Compute Architectures. 05/01/2014–-11/31/2015.

\item[3.0, co-PI] NIH 1R01NS092474-01, Synaptomes of Mouse and Man.  09/30/14--06/30/19.

\item[0.5, co-PI] NIH NIDA 1R01DA036400-01, BIGDATA: Small: DCM: ESCA: DA: Computational infrastructure for massive neuroscientific image stacks. 03/15/13--03/14/16.
\end{list}

\para{Carey E.~Priebe} received the BS degree in mathematics from Purdue University in 1984, the MS degree in computer science from San Diego State University in 1988, and the PhD degree in information technology (computational statistics) from George Mason University in 1993. From 1985 to 1994 he worked as a mathematician and scientist in the US Navy research and development laboratory system. Since 1994 he has been a professor in the Department of Applied Mathematics and Statistics, Whiting School of Engineering, Johns Hopkins University, Baltimore, Maryland. At Johns Hopkins, he holds joint appointments in the Department of Computer Science, the Department of Electrical and Computer Engineering, the Center for Imaging Science, the Human Language Technology Center of Excellence, and the Whitaker Biomedical Engineering Institute. He is a past President of the Interface Foundation of North America - Computing Science \& Statistics, a past Chair of the American Statistical Association Section on Statistical Computing, a past Vice President of the International Association for Statistical Computing, and on the editorial boards of Journal of Computational and Graphical Statistics, Computational Statistics and Data Analysis, and Computational Statistics. His research interests include computational statistics, kernel and mixture estimates, statistical pattern recognition, statistical image analysis, dimensionality reduction, model selection, and statistical inference for high-dimensional and graph data. He is a Senior Member of the IEEE, a Lifetime Member of the Institute of Mathematical Statistics, an Elected Member of the International Statistical Institute, and a Fellow of the American Statistical Association. Professor Priebe is a Research Professor in the National Security Institute at the Naval Postgraduate School, and was named one of six inaugural National Security Science and Engineering Faculty Fellows.  He was the 2010 recipient of the Distinguished Achievement Award from the American Statistical Association Section on Statistics in Defense and National Security.  He is a member of the NSA Advisory Board Mathematics Panel, and holds various security clearances.

\vspace{10pt}

\subpara{Levels of Effort}  Dr. Priebe will commit 2 months (17\%) of effort to this grant per year.  He has the following sources of funding, noted with months of effort and project role.
\small

\vspace{-5pt}

\begin{list}{[\arabic{enumi}]}{\settowidth{\labelwidth}{0.72 mos., co-PI}\leftmargin\labelwidth\advance\leftmargin 0.2in \setlength{\labelsep}{0.2in}}
\addtolength{\itemsep}{-4pt}

\item[1.0, PI]  DARPA FA8750-12-2-0303, Fusion and Inference from Multiple and Massive Disparate Distributed Dynamic Data Sets.  09/10/12 –- 03/09/17.

\item[2.0, co-PI] DOD H9823007C0365, Human Language Technology Center of Excellence. 01/13/07 -– 01/12/14. 

\item[1.0, PI] NSF DBI-1451081, Brain Eager: Discovery and Characterization of Neural Circuitry from Behavior. 09/01/14 –- 08/31/16.

\item[0.25, co-PI] DARPA N66001-14-1-4028, Scalable Brain Graph Analyses using Big-Memory High-IOPS Compute Architectures. 05/01/2014–-11/31/2015.

\end{list}

\para{Randal Burns} builds the high-performance, scalable data systems that allow scientists 
to make discoveries through the exploration, mining, and statistical analysis
of big data.  These include the Open Connectome Project (\url{openconnecto.me})  \cite{Burns13} and the
JHU Turbulence Databases (\url{turbulence.pha.jhu.edu}) \cite{Li08}.  His research contributions tear down the barriers
to using massive amounts of data either by making data access more efficient 
or improving the performance of I/O and memory systems \cite{Kanov11,Zheng13}.
% 
Burns' research approach embeds his group (students, postdocs, and programmers) in multi-disciplinary 
research teams with domain scientists 
in order to create the data systems and analysis tools that they use on 
daily basis.  This approach ensures that research results 
create new analysis capabilities that transform scientists' ability
to extract knowledge from data. 
For example, Eyink et al.~\cite{Eyink13} exploited a parallel database search to show that a 70-year-old belief about
high-conductivity plasmas---magnetic flux freezing---fails in the presence of MHD turbulence,
explaining why solar flares can erupt in minutes or hours rather than the millions of years predicted by flux freezing.

\vspace{10pt}
\subpara{Levels of Effort}  Burns will commit 2 months (17\%) of effort to this grant per year.  He has the following sources of funding, noted with months of effort and project role.
\small
\vspace{-5pt}
\begin{list}{[\arabic{enumi}]}{\settowidth{\labelwidth}{0.72 mos., co-PI}\leftmargin\labelwidth\advance\leftmargin 0.2in \setlength{\labelsep}{0.2in}}
\addtolength{\itemsep}{-4pt}

\item[1.0, PI] NIH PAR-12-806 (NIBIB), CRCNS: Data Sharing: The EM Open Connectome Project. 
09/01/12--08/31/15.

\item[0.72, co-PI] CIF21 DIBBs: Long Term Access to Large Scientific Data Sets: The SkyServer and Beyond. 10/01/13--09/30/18.

\item[1.0, PI] DARPA N66001-14-1-4028, Scalable Brain Graph Analyses using Big-Memory High-IOPS Compute Architectures. 05/01/2014–-11/31/2015.

\item[1.0, PI] NIH 1R01NS092474-01, Synaptomes of Mouse and Man.  09/30/14--06/30/19.

\item[0.12, co-PI] NIH NIDA 1R01DA036400-01, BIGDATA: Small: DCM: ESCA: DA: Computational infrastructure for massive neuroscientific image stacks. 03/15/13--03/14/16.

\end{list}



\para{Youngser Park} received the B.E. degree in electrical engineering from Inha University in Seoul, Korea in 1985, the M.S. and Ph.D. degrees in computer science from The George Washington University in 1991 and 2011 respectively. From 1998 to 2000 he worked at the Johns Hopkins Medical Institutes as a senior research engineer. From 2003 until 2011 he worked as a senior research analyst, and has been an associate research scientist since 2011 in the Center for Imaging Science at the Johns Hopkins University. At Johns Hopkins, he holds joint appointments in the Department of Applied Mathematics and Statistics and the Human Language Technology Center of Excellence.
% 
He has reviewed papers for ACM Transactions on Knowledge Discovery in Data, Statistical Analysis and Data Mining, and WIREs Computational Statistics. His current research interests are clustering algorithms, pattern classification, and data mining for high-dimensional and graph data.


% \jason{ please fix this}

\vspace{10pt}
\subpara{Levels of Effort}  Dr. Park will commit 6 months (50\%) of effort to this grant per year.  He has the following sources of funding, noted with months of effort and project role.
\small
\vspace{-5pt}
\begin{list}{[\arabic{enumi}]}{\settowidth{\labelwidth}{0.72 mos., co-PI}\leftmargin\labelwidth\advance\leftmargin 0.2in \setlength{\labelsep}{0.2in}}
\addtolength{\itemsep}{-4pt}

\item[1.0, PI] DARPA FA8750-12-2-0303, Fusion and Inference from Multiple and Massive Disparate Distributed Dynamic Data Sets.  09/10/12 –- 03/09/17.

\item[0.72, co-PI] DOD H9823007C0365, Human Language Technology Center of Excellence. 01/13/07 -– 01/12/14. 

\item[1.0, PI] NSF DBI-1451081, Brain Eager: Discovery and Characterization of Neural Circuitry from Behavior. 09/01/14 –- 08/31/16.

\end{list}


\para{Nam Lee}  received the B.A., M.S., and Ph.D. degrees in mathematics 
from the University of California, San Diego, CA, USA, 
in 2002, 2004, and 2008, respectively.
He is currently an Assistant Research Professor  in
the Department of Applied Mathematics and Statistics, Whiting School of 
Engineering, the Johns Hopkins University, Baltimore, MD, USA. 
His research interests 
include stochastic processes, 
statistical pattern recognition, 
and their applications to network science.

% \jason{ please fix this too}
\vspace{10pt}
\subpara{Levels of Effort} Dr. Nam Lee will commit 3 months (25\%) of effort to this grant per year. He has the following source of funding, noted with months of effort and project role.
\small
\vspace{-5pt}
\begin{list}{[\arabic{enumi}]}{\settowidth{\labelwidth}{0.72 mos., co-PI}\leftmargin\labelwidth\advance\leftmargin 0.2in \setlength{\labelsep}{0.2in}}
\addtolength{\itemsep}{-4pt}

\item[4.0, co-PI] DOD H9823007C0365, Human Language Technology Center of Excellence. 01/13/07 -– 01/12/14. 

\end{list}


