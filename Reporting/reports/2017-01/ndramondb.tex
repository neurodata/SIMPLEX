\documentclass[simplex.tex]{subfiles}
% NO NEED TO INPUT PREAMBLES HERE
% packages are inherited from simplex.tex; you can compile this on its own

\onlyinsubfile{
\title{NeuroData SIMPLEX Report: Subfile}
}

\begin{document}
\onlyinsubfile{
\maketitle
\thispagestyle{empty}

The following report documents the progress made by the labs of Randal~Burns and Joshua~T.~Vogelstein at Johns Hopkins University towards goals set by the DARPA SIMPLEX grant.

%%%% Table of Contents
\tableofcontents

%%%% Publications
\bibliographystyle{IEEEtran}
\begin{spacing}{0.5}
\section*{Publications, Presentations, and Talks}
\vspace{-20pt}
\nocite{*}
{\footnotesize	\bibliography{simplex}}
\end{spacing}
%%%% End Publications
}

\subsection{ndramondb}

Several new RAMON metadata queries were added to the ndstore RESTful
interfaces; these include get bounding box, query by key, and top keys.
The get bounding box query takes a RAMON metadata object ID and a base
resolution as arguments, and retrieves the three-dimensional bounding
box in voxel space for the specified object. Users can retrieve the
extent of a RAMON backed annotation and easily locate the object in 3D
space. 


Query by key allows a user to filter available RAMON objects by
specifying a key-value pair. For example, all RAMON objects of a
specific type can be queried by specifying \verb+ann\_type+ as the key and
an integer identifying the annotation type as the value. The query
returns a list of RAMON objects, which can be further queried for more
specific information (e.g. using the bounding box query above). 


The RAMON metadata standard is designed for arbitrary key/value
combinations. The top keys query allows a user to get the top K keys in
a database, where K is a user supplied parameter. The results from the
top key query can be used to inform a call to query by key, allowing a
user to explore both the RAMON metadata in a database and the available
information encapsulated within each RAMON object.


New Webservices that return JSON formatted object describing metadata
were implemented for all RAMON metadata queries.  The will be preferred
to the existing interfaces that return HDF5 objects for Web
applications.  JSON is a simple text serialization format that is widely
supported.  It removes the HDF5 software dependency to access
neuroscience metadata. HDF5, while powerful, is a cumbersome dependency
in many operating systems and frameworks.


The JSON Web services were specifically designed for the ndviz
visulation tool, which is built on Javascript.  Ndviz now supports
clickable metadata for any annotated location, i.e. a marked region in
an annotation project.  Clicking an annotation brings up the type of
annotation (neuron, axon, dendrite, synapse) properties of the
annotation (weight, confidence), and other metadata (author, creation
date).


\end{document}
