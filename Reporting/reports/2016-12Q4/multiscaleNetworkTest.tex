\documentclass[simplex.tex]{subfiles}
% NO NEED TO INPUT PREAMBLES HERE
% packages are inherited; you can compile this on its own

\onlyinsubfile{
\title{NeuroData SIMPLEX Report: Subfile}
}

\begin{document}
\onlyinsubfile{
\maketitle
\thispagestyle{empty}

The following report documents the progress made by the labs of Randal~Burns and Joshua~T.~Vogelstein at Johns Hopkins University towards goals set by the DARPA SIMPLEX grant.

%%%% Table of Contents
\tableofcontents

%%%% Publications
\bibliographystyle{IEEEtran}
\begin{spacing}{0.5}
\section*{Publications, Presentations, and Talks}
%\vspace{-20pt}
\nocite{*}
{\footnotesize	\bibliography{simplex}}
\end{spacing}
%%%% End Publications
}

\subsection{Multiscale Network Test}


To guarantee validity and consistency of MGC applied to testing in
network, we should find independent and identically distributed (i.i.d.)
configuration of each vertex in a graph (network), of which metric well
reflects the distance between vertices. We demonstrated that Euclidean
distance of raw adjacency matrix does not satisfy i.i.d assumption
generally; while diffusion maps at every time step are i.i.d under
certain latent function, which is supported by Aldous-Hoover
representation theorem and de Finette’s theorem. On the other hand,
under these theorem, graph is empty or dense. Fortunately, we have found
that exchangeable graph can be generated more generally, even containing
sparse graphs. We generate a simple simulation to check whether MGC
works or not. Thus we are going to test independence between diffusion
maps at each time point $t$ and nodal attribute $X$. 

For simulation, Stochastic Block Model (SBM) and additive and multiplicative network model have been explored which also exhibit non-linear dependence properties. What MGC does in this case is 

\begin{figure}[h!]
\begin{cframed}
\centering
\includegraphics[width=0.45\textwidth]{./figs/msnt1.png}
\includegraphics[width=0.45\textwidth]{./figs/msnt2.png}
\caption{
  hi
}
\label{fig:msnt}
\end{cframed}
\end{figure}

to test distance matrix of diffusion maps and nodal attributes,
considering $(k,l)$  nearest neighbors in each. Thus if there exists
local dependency structures or nonlinearity, the optimal neighborhood
choice of $(k,l)$ would not count every node in network in compute
distance correlation matrix. The above figures illustrate power maps of
three-block stochastic block model in terms of - nearest neighbor choice
in terms of network diffusion maps and $k$-nearest neighbor choice in
terms of nodal attributes. You can also notice that diffusion time
matters in testing. We also demonstrated that testing power of MGC
applied to diffusion maps is higher in SBM and also degree-corrected
SBM, compared to dCov, Heller-Heller-Gorfine, and latent factor network
test proposed by Fosdick and Hoff (2015).

\end{document}
