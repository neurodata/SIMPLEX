\documentclass[simplex.tex]{subfiles}
% NO NEED TO INPUT PREAMBLES HERE
% packages are inherited from simplex.tex; you can compile this on its own

\onlyinsubfile{
\title{NeuroData SIMPLEX Report: Subfile}
}

\begin{document}
\onlyinsubfile{
\maketitle
\thispagestyle{empty}

The following report documents the progress made by the labs of Randal~Burns and Joshua~T.~Vogelstein at Johns Hopkins University towards goals set by the DARPA SIMPLEX grant.

%%%% Table of Contents
\tableofcontents

%%%% Publications
\bibliographystyle{IEEEtran}
\begin{spacing}{0.5}
\section*{Publications, Presentations, and Talks}
\vspace{-20pt}
\nocite{*}
{\footnotesize	\bibliography{simplex}}
\end{spacing}
%%%% End Publications
}


\subsection{ndreg}

The Large Deformation Diffeomorphic Metric Mapping (LDDMM) algorithm is
an image registration method used to compute a smooth invertible
transform $\phi_{10}$ which aligns template image $I_0$ to target
image $I_1$. The log Jacobian determinant of the mapping $\log
|D\phi_{10}|$ measures local volume change during LDDMM.  Wherever it’s
negative the template CLARITY brain expanded to match the targeted Allen
Reference Atlas (ARA).  Wherever it’s positive the CLARITY brain
contracted.  The figure below shows the $\log |D\phi_{10}|$ overlaid
on a CLARITY brain and a histogram of its values.  It’s clear that this
brain expanded in most places to match the ARA.  This was also the case
for 8 of 9 of the of the other brains.  Our calculations showed that
CLARITY brains were 21\% smaller than the ARA on average.  This likely
occurred due to shrinkage introduced by the CLARIfying processing.


\begin{figure}[h!]
\begin{cframed}
\centering
\includegraphics[width=.73\textwidth]{./figs/ndreg.png}
\includegraphics[width=0.25\textwidth]{./figs/ndreg-md.png}\\
\includegraphics[width=0.75\textwidth]{./figs/ndreg-time.png}
\caption{
  The metric distance between the CLARITY and ARA was computed at the
  end point of the registration by integrating the time varying velocity
  fields from LDDMM.  It is a measure of how much the CLARITY brain
  differed from the ARA. The next figure compares the calculated
  distance when a Mean Square Error (MSE) cost is used during LDDMM to
  registration with Mutual Information (MI).  The metric distance was
  consistently smaller for MI under all 3 conditions (Control, Cocaine
  and Fear).  This indicates that the quality of the registration was
  higher than under a MSE cost.
}
\label{fig:ndreg}
\end{cframed}
\end{figure}

NeuroData’s registration module (ndreg) uses the Large Deformation
Diffeomorphic Metric Mapping (LDDMM) for image alignment.  LDDMM
computes a smooth invertible mapping between template image $I_0$ and
target image $I_1$. The plot below shows the timing results of experiments
registering a CLARITY brain template to the Allen Reference Atlas (ARA)
target.  In the first experiment the CLARITY image was registered to the
(ARA) using a single-scale approach on a 50 $\mu$~m grid.  In the second
experiment registration was done by a coarse to fine multi-scale method.
Registration that was done at a lower resolution was used to initialize
the algorithm at the subsequent higher resolution level.  Resolution
levels of 800, 400, 200, 100 and then 50 μm were used. The Mutual
Information (MI) between the deformed CLARITY and ARA images was
normalized to a range of $[0, 1.0]$ where 1.0 indicates that no
registration occurred and 0 indicates the best possible scenario ($I_0 =
I_1$).  The plot in figure \ref{fig:ndgeg-time} shows that the multi-scale optimization was much faster
than the single-scale method.


\begin{figure}[h!]
\begin{cframed}
\centering
\includegraphics[width=0.75\textwidth]{./figs/ndreg-single.png}
\includegraphics[width=0.75\textwidth]{./figs/ndreg-multiscale.png}
\label{fig:ndreg-scale}
\caption{
  The images below are checkerboard composite images of the deformed
  CLARITY image and the ARA.  They show that usage of the multi-scale
  approach did not reduce registration quality when compared to the
  single-scale method.
}
\end{cframed}
\end{figure}

\end{document}
