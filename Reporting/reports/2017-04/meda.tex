\documentclass[simplex.tex]{subfiles}
\definecolor{jgreen}{HTML}{197300}
\definecolor{jblue}{HTML}{0000cd}
\definecolor{jred}{HTML}{cc0000}
% NO NEED TO INPUT PREAMBLES HERE
% packages are inherited; you can compile this on its own
\definecolor{jgreen}{HTML}{197300}
\definecolor{jblue}{HTML}{0000cd}
\definecolor{jred}{HTML}{cc0000}
\begin{document}
\subsection[meda]{meda \href{https://github.com/mrae}{@JesseLP}}
%% Jan
Matrix Exploratory Data Analysis (meda) is a package being developed to
allow for easy generation of modern summary statistics effective for
high-dimensional data analysis. 

\begin{compactitem}
  \item Source code: \href{https://github.com/neurodata/meda}{https://github.com/neurodata/meda}
  \item Example output generated from Fisher's Iris data is here:
    \href{http://docs.neurodata.io/meda}{http://docs.neurodata.io/meda}
\end{compactitem}

The goal of this package is to realize the following checklist: Given a new set of n samples of vectors in $\mathbb{R}^d$

\begin{compactenum}
  \item histogram of feature types (binary, integer, non-negative, character, string etc.)
  \item \# NaNs per row? Per column? Infs per row? Per column? ``Zero'' variance rows? columns?
  \item Heat map of raw data that fits on screen (k-means++ to select 1000 samples, CUR to select 100 dimensions)
  \item 1st moment statistics
  \begin{compactenum}
    \item mean (line plot + heatmap)
    \item median (line plot + heatmap)
  \end{compactenum}
  \item 2nd moment statistics
  \begin{compactenum}
    \item correlation matrix (heatmap)
    \item matrix of energy distances (heatmap)
  \end{compactenum}
  \item density estimate
  \begin{compactenum}
    \item 1D marginals (Violin + jittered scatter plot of each dimension,  if n > 1000 or d>10, density heatmaps)
    \item 2D marginals (Pairs plots for top ~8 dimensions, if n*d>8000, 2D heatmaps)
  \end{compactenum}
  \item Outlier plot 
  \item cluster analysis (IDT++)
  \begin{compactenum}
    \item BIC curves
    \item mean line plot
    \item covariance matrix heatmaps
  \end{compactenum}
  \item spectral analysis
  \begin{compactenum}
    \item cumulative variance (with elbows) of data matrix
    \item eigenvectors (pairs plot + heatmap)
  \end{compactenum}
\end{compactenum}


\begin{compactitem}
\item To rescale the data in case of differently scaled features, we will implement the following options: 
\begin{compactitem}
  \item raw
  \item linear options
  \begin{compactitem}
    \item linear squash between 0 \& 1
    \item mean subtract and standard deviation divide
    \item median subtract and median absolute deviation divide
    \item make unit norm
  \end{compactitem}
  \item nonlinear
  \begin{compactitem}
    \item rank
    \item sigmoid squash
  \end{compactitem}
\end{compactitem}

\item To robustify in the face of outliers, we will utilize
 \href{http://projecteuclid.org/euclid.bj/1438777595}{Geometric median and robust estimation in Banach spaces} 

\item { if features have categories}
\begin{compactenum}
  \item sort by category
  \item color code labels by category
\end{compactenum}

\item { if points have categories}: 
   label points in scatter plots by symbol
\end{compactitem}

\vspace{12pt}

For point 6 (a) in the above checklist we have developed functionality
in \verb+meda+ to plot 1-dimensional heatmaps.  The 1D heatmap is
a different representation of a histogram, using color to denote count
instead of bin height, see figure~\ref{fig:meda201701}.

\begin{figure}[!h]
\begin{cframed}
\centering
\includegraphics[width=0.95\textwidth]{../../figs/201701-meda-1dheat.pdf}
\caption{A one dimensional heatmap for each of the $\log_{10}$
  transformed feature columns of the Kristina15 synaptome dataset.
  Colors correspond to the count of data points falling in to each bin.
  Scott's binning streategy is used which, in this case, yields 105 bins
  of equal width $w = 0.05$. 
  Label colors and groups starting from the bottom:
  \textcolor{jgreen}{Synap1\_F0 - Synapo\_F0 (green, excitatory)},
  \textcolor{jred}{gad\_F0 - GABABR\_F0 (red, inhibatory)},
  \textcolor{jblue}{Vglut3\_F0 - DAPI\_F0 (blue, other)}.
  }
\label{fig:meda201701}
\end{cframed}
\end{figure}

\clearpage

%%% Feb

Updates in \href{https://github.com/neurodata/meda}{meda} include the
addition of plots that explore the clusters generated by
hierarchical clustering.  We are using 
\href{http://www.stat.washington.edu/fraley/mclust/}{mclust}
in our
hierarchical clustering function.  At each level we use the Bayseian
Information Criterion (BIC) to determine if the data should be split
into two clusters or kept as one.  The dendrogram 
(figure \ref{fig:meda201702}~left)
shows the binary tree clustering structure with branch size denoting the
size of the cluster.  The stacked level means plot
(figure \ref{fig:meda201702}~right) shows the means of features in each node 
of the tree.


%\begin{figure}[!h]
%\begin{cframed}
%\centering
%\includegraphics[width=0.45\textwidth, clip = true, trim = 2cm 5cm 4cm 6cm ]{../../figs/K15_samp1e4_01e3_dendro.png}
%\includegraphics[width=0.45\textwidth, clip = true, trim = 1cm 0 5mm 1mm]{../../figs/K15_samp1e4_01e3_slcmeans.png}
%\caption{Left:  A dendrogram showing the results of hierarchical mclust.
%The splits are constrained to be binary and branch sizes show relative
%cluster sizes.  Right: A stacked level means plot showing for each node
%in the dendrogram the feature means.}
%\label{fig:meda}
%\end{cframed}
%\end{figure}
%\clearpage

%% March 

Updates:  Colored markers indicating whether BIC suggests $K = 2$ (green
triangle)  or to stop with $K = 1$ (red squares) have been added to the
dendrograms. 

\begin{figure}[!h]
\begin{cframed}
\centering
\includegraphics[width=0.45\textwidth, height = 2.5in, clip = true, trim = 0 -4cm 0 0]{../../figs/meda_dend_201703.png}
\includegraphics[width=0.45\textwidth, clip = true, trim = 1cm 0 5mm 1mm]{../../figs/meda_stacked_201703.png}
\caption{Left:  A dendrogram showing the results of hierarchical mclust.
The splits are constrained to be binary and branch sizes show relative
cluster sizes.  Right: A stacked level means plot showing for each node
in the dendrogram the feature means.}
\label{fig:meda201702}
\end{cframed}
\end{figure}

The internals of the meda package are now being re-worked so that data
processing and plotting of results will be separate. This will allow for
better modularity and faster turn-around. 

\clearpage

\end{document}
