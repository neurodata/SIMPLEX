\documentclass[simplex.tex]{subfiles}
% DO NOT INCLUDE PREAMBLES/PACKAGES HERE!!
% packages are inherited from preamble.tex; you can compile this on its own
\begin{document}
\subsection{Graph-testing}

In neuroimaging connectomics studies, it is often desired to determine
whether the observed network properties are statistically significant or
not. In order to correctly achieve this, we need to define a null
distribution. In order to generate the null distribution, the common
technique is to  generate an average of 1000 samples of graphs with the
same degree sequence of the observed graphs. However, this technique
does not yield a uniform sample from the null distribution, resulting in
ill-conditioned tests. Here, we investigate statistically accurate
methods in graph testing. \\


One strategy that we have explored is generating the null distribution
by sampling from graphs with the same degree-sequence of the observed
graph. While the samples are not generated uniformly, we know how to
rescale the samples such that we can estimate the mean of the uniform
distribution of the graphs. We are currently working on to extend this
algorithm to validly estimate the 95 percentile as well. \\


Another technique that we have explored is using parametric bootstrap to
obtain the critical region for any significance level. Namely, we fit a
stochastic block model to the data and use Generalized Likelihood Ratio
Test to determine the number of blocks that best fit the data. In the
initial experiments, as expected, we see that the power of the test
increases as the graph size increase. As the next step, after finding
the model that best first the data, we can sample from that distribution
many times, compute the test statistic, and get the critical region for
any significance level. In this strategy we are not conditioning the
graphs on their degree sequence, which is an advantage as the graphs
tend to include noise so we cannot use the observed degree sequence as
the ground truth. 

%%%   EXAMPLE FIGURE BLOCK
%\begin{figure}[!h]
%\begin{cframed}
%\centering
%\includegraphics[width=0.15\textwidth]{../../figs/neurodata_small.png}
%\caption{Please provide a detailed caption for your figure.}
%\label{fig:name}
%\end{cframed}
%\end{figure}

\clearpage
\end{document}
