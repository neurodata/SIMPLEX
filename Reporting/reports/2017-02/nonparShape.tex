\documentclass[simplex.tex]{subfiles}
% NO NEED TO INPUT PREAMBLES HERE
% packages are inherited; you can compile this on its own
\begin{document}
\subsection{Non-Parametric Shape Clustering}
%
%We developed a prototype algorithm for 2-dimensional shape clustering, which
%is invariant under affine transformations. This employs the procrustes
%distance between two objects, which requires feature extraction to obtain
%landmark points; see Fig.~\ref{fig:nonpar} for an example. 
%We intend to apply a variant of this method to analyze neural synapses,
%since we have such datasets from our collaborators. However, to this particular
%dataset, the preprocessing techniques may be highly sophisticated, and a new
%metric to compare different objects may be necessary. Moreover, these shapes are
%3-dimensional. We are currently working on this project
%with our collaborators, extending our existing  techniques to this 
%particular dataset.
%
%We also started working on a related, and more general, project.
%We intend to develop non-parametric clustering algorithms with statistical
%guarantees. We will use an energy-statistics based approach. Given
%two datasets $X$ and $Y$, there is an energy function $\mathcal{E}(X,Y)$
%test statistic which allows us to infer if $X$ and $Y$ have the same
%distribution. Our results thus far suggest that this can be written
%as a quadratic optimization problem
%with quadratic constraints: 
%\begin{equation}
%\max_{x,z\in \mathbb{R}^N} x^T \Delta z \qquad \mbox{s.t. $x_i^2=1$, $x+z = 0$}
%\end{equation}
%where $\Delta$ is a dissimilarity data matrix.
%There is not enough literature on
%this interesting problem, so this will very likely lead to new methods which
%can have interesting applications, in particular to neuroscience datasets.
%
%\begin{figure}[h!]
%\begin{cframed}
%\centering
%\begin{subfigure}[t]{0.45\textwidth}
%\includegraphics[width=\textwidth]{../../figs/nonpar57.png}
%\label{fig:nonpar57}
%\caption{
%  MNIST digits with extracted landmarks and alignment.
%  }
%\end{subfigure}
%\begin{subfigure}[t]{0.45\textwidth}
%\includegraphics[width=\textwidth]{../../figs/nonPar357.png}
%\label{fig:nonpar357}
%\caption{
%Classification error against the size of each
%cluster (the three classes have the same number of points) is
%shown in blue. The red line is standard
%K-means with Euclidean distance for comparison.}
%\end{subfigure}
%\caption{
%  MNIST handwritten digits and classification error results.
%}
%\label{fig:nonpar}
%\end{cframed}
%\end{figure}
%
\end{document}
