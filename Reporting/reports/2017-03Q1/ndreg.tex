\documentclass[simplex.tex]{subfiles}
% NO NEED TO INPUT PREAMBLES HERE
% packages are inherited from simplex.tex; you can compile this on its own
\begin{document}
\subsection{ndreg}
%
%
%The NeuroData Registration python module, \textit{ndreg}, uses \textit{Large Deformation Diffeomorphic Metric Mapping (LDDMM)} to register a template image $I_0$ to a target image $J_1$.
%It does this by finding smooth invertible map $\varphi$ such that the \textit{matching term} $M(I_0 \circ \varphi^{-1}, J_1)$, a function whose value is small when $I_0 \circ \varphi^{-1}$ is aligned with $J_1$, is minimized.
%We compared MI-LDDMM, LDDMM with a Mutual Information based matching term, to our previous SSD-LDDMM and Mask-LDDMM techniques.
%SSD-LDDMM uses a Sum of Squared Differences matching term.
%As it is based on image subtraction, it assumes that bright regions in $I_0$ align with bright regions in $J_1$.
%Mask-LDDMM is SSD-LDDMM in which $I_0$ and $J_1$ are replaced with their respective binary brain mask images $M_0$ and $M_1$.
%Since the masks only contain information on which voxels are inside the brain, only edge information is used by Mask-LDDMM.
%\\
%We placed fiducial landmarks in the corpus callosum and midbrain of four CLARITY template image and the \textit{Allen Reference Atlas (ARA)} target.
%After registration corresponding CLARITY landmarks should be close to those of the ARA. 
%If not then the registration was inadequate.
%Figure~\ref{fig:comparison} shows that MI-LDDMM registration performed better than the previous methods.
%
%\begin{figure}[!h]
%\begin{cframed}
%\centering
% \begin{subfigure}{0.25\columnwidth}
%  \includegraphics[width=\textwidth]{../../figs/ssdCoronal.png}  
%  \caption{SSD-LDDMM}
%  \label{fig:clarityCoronalSSD}
% \end{subfigure}
% \begin{subfigure}{0.25\columnwidth}
%  \includegraphics[width=\textwidth]{../../figs/maskCoronal.png}  
%  \caption{Mask-LDDMM}
%  \label{fig:clarityCoronalMask}
% \end{subfigure}
% \begin{subfigure}{0.25\columnwidth}
%  \includegraphics[width=\textwidth]{../../figs/miCoronal.png}  
%  \caption{MI-LDDMM}
%  \label{fig:clarityCoronalMI}
% \end{subfigure}
% \begin{subfigure}{0.2\columnwidth}
%  %\captionsetup{justification=centering} % Center caption
%  \includegraphics[width=\textwidth]{../../figs/lmkError.png}  
%  \caption{Landmark Error}
%  \label{fig:lmkError}
% \end{subfigure}
% \caption{Comparison of SSD-LDDMM (\subref{fig:clarityCoronalSSD}), Mask-LDDMM (\subref{fig:clarityCoronalMask}) and MI-LDDMM (\subref{fig:clarityCoronalMI}) registration for a CLARITY mouse brain.
%  Panes (\subref{fig:clarityCoronalSSD}-\subref{fig:clarityCoronalMI}) have an ARA coronal slice on the left juxtaposed to the corresponding aligned CLARITY slice on the right.
%  Green arrows point out that the corpus callosum is misaligned by SSD-LDDMM but aligned correctly by MI matching.
%  Red arrows show that SSD-LDDMM distorts bright regions.
%  Fiducial landmarks were placed throughout the corpus callosum, and midbrain of the acquired volumes.
%  Pane (\subref{fig:lmkError}) compares mean errors between the deformed CLARITY and ARA landmarks after registration.
% }
% \label{fig:comparison}
%\end{cframed}
%\end{figure}
%
%\clearpage


We received a rat brain image acquired using iDISCO microscopy from colleagues at the Johns Hopkins School of Medicine. 
iDISCO is a technique for clearing tissues that enables interrogatoion by light-sheet microscopy. 
Thus like CLARITY, iDISCO cleared brains can be imaged at high spatial resolution without physical slicing.
The iDISCO volume was ingested into the NeuroData store at its full resolution of 5 $\mu$m isotropic.
We also ingested the widely used Waxholm Rat Atlas, a T2 magnetic resonance image with corresponding labels at a 39 $\mu$m isotropic.\\

\textit{Large Deformation Diffeomorphic Metric Mapping (LDDMM)} is a deformable image registration (alignment) algorithm which computes smooth invertible transforms between images.
Using the \textit{NeuroData Registration (ndreg)} python module we were able to perform a iDISCO to T2 MRI alignment by LDDMM under mutual information matching.
Figure~\ref{fig:idisco} shows the Waxholm Rat Atlas labels overlaid on the iDISCO image.

\begin{figure}[!h]
 \begin{cframed}
  \centering
   \includegraphics[width=\textwidth]{../../figs/iDISCO.png}  
   \caption{Waxholm rat atlas labels overlaid on sagittal slices of iDISCO rat brain}
  \label{fig:idisco}
 \end{cframed}
\end{figure}

\end{document}
