\documentclass[simplex.tex]{subfiles}
% NO NEED TO INPUT PREAMBLES HERE
% packages are inherited; you can compile this on its own
\begin{document}
\subsection[meda]{meda \href{https://github.com/mrae}{@JesseLP}}
%
%Updates in \href{https://github.com/neurodata/meda}{meda} include the
%addition of plots that explore the clusters generated by
%hierarchical clustering.  We are using 
%\href{http://www.stat.washington.edu/fraley/mclust/}{mclust}
%in our
%hierarchical clustering function.  At each level we use the Bayseian
%Information Criterion (BIC) to determine if the data should be split
%into two clusters or kept as one.  The dendrogram 
%(figure \ref{fig:meda}~left)
%shows the binary tree clustering structure with branch size denoting the
%size of the cluster.  The stacked level means plot
%(figure \ref{fig:meda}~right) shows the means of features in each node 
%of the tree.
%
%
%\begin{figure}[!h]
%\begin{cframed}
%\centering
%\includegraphics[width=0.45\textwidth, clip = true, trim = 2cm 5cm 4cm 6cm ]{../../figs/K15_samp1e4_01e3_dendro.png}
%\includegraphics[width=0.45\textwidth, clip = true, trim = 1cm 0 5mm 1mm]{../../figs/K15_samp1e4_01e3_slcmeans.png}
%\caption{Left:  A dendrogram showing the results of hierarchical mclust.
%The splits are constrained to be binary and branch sizes show relative
%cluster sizes.  Right: A stacked level means plot showing for each node
%in the dendrogram the feature means.}
%\label{fig:meda}
%\end{cframed}
%\end{figure}
%
%\clearpage
\end{document}
